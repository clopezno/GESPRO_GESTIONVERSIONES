\capitulo{2}{Objetivos del proyecto} \label{sec:objetivos}

\section{Objetivos funcionales}

Estos objetivos se centran en las funcionalidades y características que debe tener la aplicación \textit{Eco City Tours} para satisfacer las necesidades y expectativas de los usuarios. A continuación se detallan los objetivos funcionales del proyecto:

\begin{itemize}
    \item \textbf{Propuesta de rutas turísticas personalizadas}: La aplicación debe ser capaz de generar rutas turísticas personalizadas basadas en las preferencias del visitante utilizando \acrfull{llm}. Para llevarlo a cabo, el usuario facilitará al modelo sus preferencias, eligiendo entre otras opciones el medio de transporte elegido o el número de \acrfull{pdi} a visitar.
    \item \textbf{Obtener los \acrfull{pdi}}: A través de la interacción con el modelo \acrshort{llm}, la aplicación le indicará que debe priorizar un \acrshort{pdi} sobre otro en función de criterios sostenibles como la deslocalización del turismo y preferencias de usuario como puedan ser duración de la visita o medio de transporte ecológico a utilizar.
    \item \textbf{Visualización de rutas en mapa}: La aplicación debe mostrar las rutas sugeridas en un mapa utilizando herramientas \acrshort{sig}.
    \item \textbf{Gestión de rutas}: La aplicación permitirá a los usuarios crear, guardar y cargar rutas turísticas, facilitando una mayor personalización y aprovechamiento de la experiencia turística.
    

\end{itemize}

\section{Objetivos no funcionales}

Los objetivos no funcionales se refieren a los desafíos y metas que se deben abordar para desarrollar el software. Estos objetivos abarcan aspectos como la arquitectura del sistema, las tecnologías a utilizar y las metodologías de desarrollo. A continuación se detallan los objetivos no funcionales del proyecto:

\begin{itemize}
	\item \textbf{Integración de inteligencia artificial usando \acrfull{nlp}}: La aplicación contará con un modelo de lenguaje preseleccionado, que facilita la propuesta de rutas turísticas personalizadas y la obtención de información relevante de \acrlong{pdi}. Esta elección garantiza la estabilidad y el rendimiento de la aplicación, proporcionando información precisa y relevante sin la necesidad de cambiar modelos, lo cual reduce la complejidad de mantenimiento.
    \item \textbf{Usabilidad}: la interacción del usuario con la aplicación debe ser intuitiva y sencilla, permitiendo un rápido aprendizaje de todas sus funcionalidades. El diseño de la interfaz debe estar orientado a ofrecer una experiencia de uso fluida. 
\end{itemize}

\section{Objetivos personales}

\begin{itemize}
	\item \textbf{Formación en \acrshort{llm} y su integración en aplicaciones software.} Dada la rápida evolución de los \acrfull{llm} y la amplitud de campos del conocimiento en los que se pueden utilizar, obtener una base de conocimientos destacable en este área sería un objetivo que me permitiría expandir mi futuro académico y por tanto distinguir mi perfil profesional especializándome en un sector con fuerte expansión.
	\item \textbf{Desarrollo de aplicación móvil profesional}: poner en práctica lo aprendido en varios cursos de auto-formación online en \textbf{Dart y Flutter.} La aplicación de este proyecto puede ser parte de mi porfolio con aplicaciones que muestren mis habilidades a futuros empleadores.
	\item \textbf{Finalización del \acrshort{tfg} y Grado}: tras no haber completado la Ingeniería Técnica Informática en su momento por no haber realizado el Proyecto Fin de Carrera, la realización de este \acrshort{tfg} marca la culminación de mi formación académica como ingeniero.
\end{itemize}
